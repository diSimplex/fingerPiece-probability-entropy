% !TEX root = probabilityEntropy.tex
% !LPiL preamble = ./pePreamble.tex
% !LPiL postamble = ./pePostamble.tex
% !LPiL collection = fingerPieces

\lpilTitle{fp-probEnt}[
  Finger Pieces : Probability and Entropy
]{
  Finger Pieces : A look at Probability and Entropy.
}
\author{Stephen Gaito}

\maketitle

\begin{abstract}
  In this finger piece, we explore the Probability and Entropy as used in the
  diSimplex project.
\end{abstract}

We investigate the definition of Probability and Entropy as extensions of Logic
as expounded by Cox (\cite{cox1962algProbableInference}), and Jaynes
(\cite{jaynes1994probTh}).

We will actually follow the analysis and axioms provided in
\cite{arnborgSjödin2000bayesRulesFiniteModels} and
\cite{dupréTipler2009axiomsBayesianProb}, since their analysis and axioms
provide definitions which do not require continuity nor differentiability
and so work for finite structures. These works also make use of a
``coarse-graining'' approach which we will use extensively in these notes.

Both Cox and Jaynes assume classical (Boolean) Logic. In this finger piece we
extend these definitions to make use of Heyting Logic as provided as the
internal logic of any general Topos. We do this so that we can mimic Christopher
Isham's ``coarse grained'' approach to Quantum Mechanics (and eventually Quantum
Relativity). For the purposes of Probability see
\cite{isham2002reflectionsConventionalGravity}.

All of the ``classical'' approaches listed above, \emph{assume} that probability
is a Real between $0$ and $1$. However, working in a Topos,
\cite{isham2002reflectionsConventionalGravity}, strongly suggests that
probability should be an element of the ``internal Reals''. In our case these
``internal Reals'' \emph{should} end up being Peter Walley's imprecise
(interval) probabilities (\cite{walley1991impreciseProb}).

Eventually we will use \cite{goyalKnuthSkilling2009complexQuantumAmplitudes} to
convert our Topos based definitions to more traditional Quantum Amplitudes.
